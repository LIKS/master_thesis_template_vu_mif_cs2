\section{Dirbtinių neuroninių tinklų apžvalga}
\label{sec:Dirbtiniu neuroniniu tinklu apzvalga}

\subsection{Bendrieji dirbtinių neuroninių tinklų principai}
\subsubsection{Perceptronas}
\label{sec:Perceptronas}
Perceptronas\footnote{Perceptronu vadinsime McCulloch-Pitts neuroną su
sigmoidine aktyvavimo funkcija.} yra iteraciškai apmokomas tiesinis
klasifikatorius. % Perceptrono atliekamų veiksmų seką galime aprašyti taip:
Įvedami žymėjimai:
duomenų vektorių $\bm{x} = (x_1, x_2, ..., x_p)$ praplečiame vienetu, $\bm{z} =
(1, x_1, x_2, ..., x_p)$, perceptrono įėjimų svorių vektorių $\bm{w} = (w_1,
w_2, ..., w_p)$ praplečiame $w_0$, $\bm{v} = (w_0, w_1, w_2, ..., w_p)$.
Naudosime sigmoidinę glodninimo funkciją:
\begin{equation} \label{eq:sigmoid}
f(x) = {1 \over 1 + e^{-x}} \;.
\end{equation}

Tada $i$-tajam duomenų vektoriui perceptrono išėjimas apskaičiuojamas taip:
\begin{equation} \label{eq:perceptron output}
o_i = f(\bm{x}_i \bm{w}^T + w_0) = f(\bm{z}_i \bm{v}^T) \;.
\end{equation}

...
