\vumifsectionnonumnocontent{Santrauka}

Praktiniuose taikymuose, kaip kad medicinos diagnostikoje ir veidų atpažinime, pripažintas santykinis klasifikavimo klaidų kainų, priklausančių nuo tikrosios ir priskirtosios klasių, skirtumas. Šis darbas apima klaidų kainoms jautraus hibridinio klasifikatoriaus, sudaryto iš sprendimo medžio ir daugiasluoksnio perceptrono, kūrimą ir analizę. Hibridiniam klasifikatoriui konstruoti buvo realizuoti kainoms jautrių \emph{C4.5} sprendimo medžių variantai bei klaidų kainoms jautrus daugiasluoksnis perceptronas, jiems kombinuoti panaudota \emph{Banerjee} metodika. Atlikti eksperimentai su realaus pasaulio ir sintetiniais duomenimis leidžia teigti, kad hibridinis klasifikatorius gali būti naudingas, t. y. sumažinti jį inicializavusio sprendimo medžio klasifikavimo kainą ir pasiekti šį rezultatą greičiau nei atsitiktiniais svoriais inicializuotas daugiasluoksnis perceptronas.

\bigskip
Raktiniai žodžiai: nesubalansuoti duomenys, jautrus kainoms, klasifikavimas, dirbtiniai neuroniniai tinklai, sprendimų medis, hibridinis, daugiasluoknis perceptronas.
