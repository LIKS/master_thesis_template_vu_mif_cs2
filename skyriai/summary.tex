\vumifsectionnonumnocontent{Summary}

In practical applications including medical diagnosis and face detection it has been admitted that classification errors might differ in relative cost depending on the real and predicted classes. The work comprises implementation and analysis of a cost-sensitive hybrid classifier, consisting of a decision tree and a multi-layer perceptron. The hybrid classifier was constructed using several varieties of a cost-sensitive \emph{C4.5} decision tree and a cost-sensitive multi-layer perceptron, which were combined using the \emph{Banerjee} method. Conducted experiments with real world and synthetic data allow to conclude that the hybrid method might be useful, namely, decrease the misclassification error cost of the initializing tree and achieve this result faster than a randomly initialized multi layer perceptron.

\bigskip
Keywords: imbalanced dataset, cost-sensitive, classification, artificial neural network, decision tree, hybrid, multilayer perceptron.
