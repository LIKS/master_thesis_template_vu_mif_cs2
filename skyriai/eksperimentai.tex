\section{Hibridinių klasifikatorių veikimo eksperimentinis tyrimas}

\subsection{Bendrieji eksperimentų nustatymai}

Eksperimentais šiame skyrelyje siekiama nustatyti, ar apskritai yra prasminga kurti kainoms jautrų hibridinį klasifikatorių, panaudojant SM ir DNT hibridizacijos metodiką \cite{Banerjee1997} ir įvedant jautrumą kainoms kiekvienoje klasifikatoriaus dalyje nepriklausomai. Kad būtų prasminga, reikėtų parodyti, kad hibridinis klasifikatorius sugeba pasiekti mažesnę klasifikavimo kainą su testiniais duomenimis nei jį inicializavęs sprendimo medis per mažiau iteracijų nei tokios pat architektūros, tačiau atsitiktiniais svoriais inicializuotas DSP.

...

\begin{table}[h!]
  \begin{center}
    \begin{tabular}{|c|p{4.5cm}|p{5cm}|p{3.5cm}|}
      \hline
      & \multicolumn{1}{|c|}{Pavadinimas} & {Inicializuojantis medis} & {Kainoms jautrus algoritmo aspektas} \\
      \hline
      1. & \emph{Hybrid\_C4.5} & C4.5 medis & DSP \\
      2. & \emph{Hybrid\_Laplace} & C4.5 medis, Laplace genėjimas & Genėjimas, DSP \\
      3. & \emph{Hybrid\_MetaCost} & MetaCost medis, naudojantis C4.5 medį kaip bazinį klasifikatorių & MetaCost, DSP \\
      4. & \emph{Hybrid\_Mod\_prob} & C4.5 medis, pakeitus apriorines klasių tikimybes ir lapų klases pagal kainų matricą & Tikimybių modifikacija ir lapų pernumeravimas, DSP \\
      5. & \emph{Hybrid\_Mod\_prob\_err} & C4.5 medis, pakeitus apriorines klasių tikimybes pagal kainų matricą, klaidomis grįstas genėjimas & Tikimybių modifikacija, DSP\\
      6. & \emph{Hybrid\_Mod\_prob\_lap} & C4.5 medis, pakeitus apriorines klasių tikimybes pagal kainų matricą, Laplaso genėjimas & Tikimybių modifikacija, genėjimas, DSP \\
      7. & \emph{Hybrid\_C5.0} & C5.0 medis & C5.0 medis, DSP \\
      8. & \emph{Banerjee} & C4.5 medis & Kainoms nejautrus \\
      \hline
    \end{tabular}
  \end{center}
  \caption{Lyginami hibridiniai klasifikatoriai.}\label{tab:implemented classifiers}
\end{table}

...
